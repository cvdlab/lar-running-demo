%% Based on a TeXnicCenter-Template by Tino Weinkauf.
%%%%%%%%%%%%%%%%%%%%%%%%%%%%%%%%%%%%%%%%%%%%%%%%%%%%%%%%%%%%%

%%%%%%%%%%%%%%%%%%%%%%%%%%%%%%%%%%%%%%%%%%%%%%%%%%%%%%%%%%%%%
%% HEADER
%%%%%%%%%%%%%%%%%%%%%%%%%%%%%%%%%%%%%%%%%%%%%%%%%%%%%%%%%%%%%
\documentclass[a4paper,twoside,10pt]{report}
% Alternative Options:
%	Paper Size: a4paper / a5paper / b5paper / letterpaper / legalpaper / executivepaper
% Duplex: oneside / twoside
% Base Font Size: 10pt / 11pt / 12pt


%% Language %%%%%%%%%%%%%%%%%%%%%%%%%%%%%%%%%%%%%%%%%%%%%%%%%
\usepackage[british]{babel} %francais, polish, spanish, ...
\usepackage[T1]{fontenc}
\usepackage[ansinew]{inputenc}
\usepackage[plainpages=false]{hyperref}	% generazione di collegamenti ipertestuali su indice e riferimenti
\usepackage[all]{hypcap} % per far si che i link ipertestuali alle immagini puntino all'inizio delle stesse e non alla didascalia 
\usepackage{lmodern} %Type1-font for non-english texts and characters


%% Packages for Graphics & Figures %%%%%%%%%%%%%%%%%%%%%%%%%%
\usepackage{graphicx} %%For loading graphic files
%\usepackage{subfig} %%Subfigures inside a figure
%\usepackage{tikz} %%Generate vector graphics from within LaTeX

%% Please note:
%% Images can be included using \includegraphics{filename}
%% resp. using the dialog in the Insert menu.
%% 
%% The mode "LaTeX => PDF" allows the following formats:
%%   .jpg  .png  .pdf  .mps
%% 
%% The modes "LaTeX => DVI", "LaTeX => PS" und "LaTeX => PS => PDF"
%% allow the following formats:
%%   .eps  .ps  .bmp  .pict  .pntg


%% Math Packages %%%%%%%%%%%%%%%%%%%%%%%%%%%%%%%%%%%%%%%%%%%%
\usepackage{amsmath}
\usepackage{amsthm}
\usepackage{amsfonts}
%
\usepackage{xcolor}
\definecolor{whitesmoke}{HTML}{F5F5F5}
%
\usepackage{listings}
\lstdefinestyle{bash-style}{
  captionpos=b,
  belowcaptionskip=1\baselineskip,
  breaklines=true,
  tabsize=2,
  frame=tb,
  aboveskip=3mm,
  belowskip=3mm,
  xleftmargin=\parindent,
  language=bash,
  showstringspaces=false,
  % basicstyle=\tiny, %\footnotesize\ttfamily,
  basicstyle={\fontsize{8pt}{8pt}\ttfamily},
  keywordstyle=\color{blue},
  commentstyle=\color{green!40!black},
  stringstyle=\color{brown},
  %identifierstyle=\color{blue},
  backgroundcolor=\color{whitesmoke}
}

%% Line Spacing %%%%%%%%%%%%%%%%%%%%%%%%%%%%%%%%%%%%%%%%%%%%%
%\usepackage{setspace}
%\singlespacing        %% 1-spacing (default)
%\onehalfspacing       %% 1,5-spacing
%\doublespacing        %% 2-spacing


%% Other Packages %%%%%%%%%%%%%%%%%%%%%%%%%%%%%%%%%%%%%%%%%%%
%\usepackage{a4wide} %%Smaller margins = more text per page.
%\usepackage{fancyhdr} %%Fancy headings
%\usepackage{longtable} %%For tables, that exceed one page

%%%%%%%%%%%%%%%%%%%%%%%%%%%%%%%%%%%%%%%%%%%%%%%%%%%%%%%%%%%%%
%% DOCUMENT
%%%%%%%%%%%%%%%%%%%%%%%%%%%%%%%%%%%%%%%%%%%%%%%%%%%%%%%%%%%%%
\begin{document}

\pagestyle{empty} %No headings for the first pages.


%% Title Page %%%%%%%%%%%%%%%%%%%%%%%%%%%%%%%%%%%%%%%%%%%%%%%
%% ==> Write your text here or include other files.

%% The simple version:
\title{Lar Running Demo}
\author{Francesco Furiani}
%\date{} %%If commented, the current date is used.
\maketitle

%% Inhaltsverzeichnis %%%%%%%%%%%%%%%%%%%%%%%%%%%%%%%%%%%%%%%
\tableofcontents %Table of contents
\cleardoublepage %The first chapter should start on an odd page.

\pagestyle{plain} %Now display headings: headings / fancy / ...



%% Chapters %%%%%%%%%%%%%%%%%%%%%%%%%%%%%%%%%%%%%%%%%%%%%%%%%
%% ==> Write your text here or include other files.

%\input{intro} %You need a file 'intro.tex' for this.
This document contains the documentation for the software <<lar-running-demo>> (LRD). 

%%%%%%%%%%%%%%%%%%%%%%%%%%%%%%%%%%%%%%%%%%%%%%%%%%%%%%%%%%%%%
%% ==> Some hints are following:

\chapter{Purpose}\label{purpose}

LRD is a software that enables automatic model extraction from a set of medical images (like MRI). Such feature is achieved through a process in which a user gives input and other parameters to start the computation.

The whole software is based on the mathematical LAR framework\cite{dicarlo2013linear} and the OpenCL based framework "HPC LAR"\cite{Furiani2013}.

\chapter{Prerequisites}\label{prerequisites}

Although the system checks for needed prerequisites, they're listed here for easy of use.

\begin{itemize}
\item *NIX compatible operating system
\item Bash
\item Pyhton (at least 2.7)
\item Java (at least 1.7) \footnotemark[1]
\item OpenCL libraries with a compatible device \footnotemark[1]
\end{itemize}
\footnotetext[1]{These are not mandatory, but if present will speed-up the computation process.}

\section{Pyhton modules}
The system is mostly wrote in Python and some modules are necessary for its execution. They can installed through your installer of choice, for example \texttt{easy\_install} or \texttt{pip}, or they might be already part of your Python distribution. Only one, requires manual installation.

\begin{enumerate}
\item NumPy
\item SciPy
\item Cython
\item Simplejson (or Json)
\item Requests
\item Termcolor
\item Png
\item Matplotlib
\item Logging
\item Multiproccessing
\item PyPlasm (\url{https://github.com/plasm-language/pyplasm})
\end{enumerate}

\section{OpenCL}
Using OpenCL requires installation of proper platform accelerated drivers and libraries. Check the documentation of your vendor to do so.

\section{Java}
A standard Java installation is needed to us OpenCL. Setting \texttt{\$JAVA\_HOME} is mandatory to ensue correct execution, especially on NVIDIA platforms.

\chapter{Using}\label{using}
The program consists of 3 bash scripts that can be used from a *NIX compatible shell:
\begin{itemize}
	\item \texttt{startConversion.sh}
	\item \texttt{startConversion-singleColor.sh}
	\item \texttt{visualize.sh}
\end{itemize}

Every script contains a \texttt{-h} option that can be used to discover the relevant options for each script as in \ref{lst:help-example}

\begin{lstlisting}[style=bash-style,label=lst:help-example]
@> ./startConversion.sh -?
===================
LAR Model extractor
===================
Either run without args or with
To use args enable it with -u
-d <input images>       : Directory containing input images
-f <best image>         : Image to quantize on
-q <colors>             : Number of colors to quantize (min 2)
-c                      : Use OpenCL
\end{lstlisting}

Command line switches are not mandatory: every script has an interactive shell in which you can input, step by step, the correct input for the program.

\section{Input}
\subsection{startConversion scripts}

\subsection{visualize}



%%%%%%%%%%%%%%%%%%%%%%%%%%%%%%%%%%%%%%%%%%%%%%%%%%%%%%%%%%%%%



%%%%%%%%%%%%%%%%%%%%%%%%%%%%%%%%%%%%%%%%%%%%%%%%%%%%%%%%%%%%%
%% BIBLIOGRAPHY AND OTHER LISTS
%%%%%%%%%%%%%%%%%%%%%%%%%%%%%%%%%%%%%%%%%%%%%%%%%%%%%%%%%%%%%
%% A small distance to the other stuff in the table of contents (toc)
\addtocontents{toc}{\protect\vspace*{\baselineskip}}

%% The Bibliography
%% ==> You need a file 'literature.bib' for this.
%% ==> You need to run BibTeX for this (Project | Properties... | Uses BibTeX)
\addcontentsline{toc}{chapter}{Bibliography} %'Bibliography' into toc
\nocite{*} %Even non-cited BibTeX-Entries will be shown.
\bibliographystyle{plain} %Style of Bibliography: plain / apalike / amsalpha / ...
\bibliography{literature} %You need a file 'literature.bib' for this.

%% The List of Figures
%\clearpage
%\addcontentsline{toc}{chapter}{List of Figures}
%\listoffigures

%% The List of Tables
%\clearpage
%\addcontentsline{toc}{chapter}{List of Tables}
%\listoftables


%%%%%%%%%%%%%%%%%%%%%%%%%%%%%%%%%%%%%%%%%%%%%%%%%%%%%%%%%%%%%
%% APPENDICES
%%%%%%%%%%%%%%%%%%%%%%%%%%%%%%%%%%%%%%%%%%%%%%%%%%%%%%%%%%%%%
\appendix
%% ==> Write your text here or include other files.

%\input{FileName} %You need a file 'FileName.tex' for this.


\end{document}

